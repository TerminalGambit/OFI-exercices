\documentclass{article}
\usepackage[utf8]{inputenc}
\usepackage{amsmath, amssymb}

\title{Fiche Récapitulative: Formules de Dénombrement}
\author{Jack Massey}
\date{}

\begin{document}

\maketitle

\section{Principes de Base}
\begin{itemize}
    \item \textbf{Principe Fondamental de Dénombrement:} Si une opération peut se faire de \(m\) manières et une autre opération peut se faire de \(n\) manières, alors les deux opérations peuvent se faire de \(m \times n\) manières.
\end{itemize}

\section{Permutations}
\begin{itemize}
    \item \textbf{Permutations de \(n\) objets distincts:} \(P(n) = n!\)
    \item \textbf{Permutations de \(n\) objets avec répétition:} \(P_{n_1, n_2, \ldots, n_k} = \frac{n!}{n_1! \cdot n_2! \cdots n_k!}\) où \(n_1, n_2, \ldots, n_k\) sont les nombres d'objets identiques.
\end{itemize}

\section{Combinaisons}
\begin{itemize}
    \item \textbf{Combinaisons de \(n\) objets pris \(k\) à la fois:} \(C(n, k) = \binom{n}{k} = \frac{n!}{k!(n-k)!}\)
\end{itemize}

\section{Arrangements}
\begin{itemize}
    \item \textbf{Arrangements de \(n\) objets pris \(k\) à la fois:} \(A(n, k) = \frac{n!}{(n-k)!}\)
\end{itemize}

\section{Formules Spécifiques}
\begin{itemize}
    \item \textbf{Nombre de façons de partitionner un entier \(n\):} Utilisation de la fonction de partition \(p(n)\).
    \item \textbf{Formule de Hardy-Ramanujan (approximation):} \(p(n) \approx \frac {1} {4n\sqrt{3}} e^{\pi \sqrt{\frac{2n}{3}}}\)
\end{itemize}

\section{Applications Pratiques}
\begin{itemize}
    \item Expliquer comment chaque formule peut être utilisée dans des situations pratiques, telles que le dénombrement de combinaisons dans un jeu, l'organisation d'objets, etc.
\end{itemize}

\end{document}
