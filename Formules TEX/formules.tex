\documentclass{article}
\usepackage[utf8]{inputenc}
\usepackage{amsmath, amssymb}

\title{Fiche Récapitulative: Formules de Dénombrement}
\author{Jack Massey}
\date{}

\begin{document}

\maketitle

\section{Principes de Base}
\begin{itemize}
    \item \textbf{Principe Fondamental de Dénombrement:} Si une opération A peut être réalisée de \(m\) manières différentes et une opération B indépendante de \(n\) manières différentes, alors il y a \(m \times n\) façons de réaliser les deux opérations.
\end{itemize}

\section{Permutations}
\begin{itemize}
    \item \textbf{Permutations de \(n\) objets distincts:} \(P(n) = n!\). Utilisée lorsque l'ordre des objets est important.
    \item \textbf{Permutations avec répétition:} Si dans un ensemble de \(n\) objets, \(n_1\) objets sont du type 1, \(n_2\) du type 2, ..., \(n_k\) du type \(k\), alors le nombre total de permutations est \(\frac{n!}{n_1! \cdot n_2! \cdots n_k!}\).
\end{itemize}

\section{Combinaisons}
\begin{itemize}
    \item \textbf{Combinaisons de \(n\) objets pris \(k\) à la fois:} \(C(n, k) = \binom{n}{k} = \frac{n!}{k!(n-k)!}\). Utilisée pour compter le nombre de façons de choisir \(k\) objets parmi \(n\), sans tenir compte de l'ordre.
\end{itemize}

\section{Arrangements}
\begin{itemize}
    \item \textbf{Arrangements de \(n\) objets pris \(k\) à la fois:} \(A(n, k) = \frac{n!}{(n-k)!}\). Similaire aux permutations, mais seulement \(k\) objets sont choisis parmi \(n\).
\end{itemize}

\section{Formules Spécifiques}
\begin{itemize}
    \item \textbf{Partition d'un entier \(n\):} Le nombre de façons de diviser \(n\) en sommes d'entiers positifs. Calculé par la fonction de partition \(p(n)\).
    \item \textbf{Formule de Hardy-Ramanujan (approximation):} \(p(n) \approx \frac {1} {4n\sqrt{3}} e^{\pi \sqrt{\frac{2n}{3}}}\). Donne une approximation pour les grands nombres.
\end{itemize}

\section{Applications Pratiques}
\begin{itemize}
    \item \textbf{Permutations:} Utilisées pour déterminer le nombre de façons différentes d'organiser des éléments. Par exemple, l'ordre des livres sur une étagère ou l'ordre des coureurs dans une course.
    \item \textbf{Combinaisons:} Utiles pour calculer le nombre de groupes possibles. Par exemple, sélectionner une équipe à partir d'un groupe plus large, ou choisir des sujets dans un questionnaire.
    \item \textbf{Arrangements:} Appliqués lorsque l'ordre est important mais que seuls quelques éléments sont choisis. Par exemple, l'attribution des premières places dans une compétition.
\end{itemize}

\end{document}
